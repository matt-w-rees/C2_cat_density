\documentclass[]{elsarticle} %review=doublespace preprint=single 5p=2 column
%%% Begin My package additions %%%%%%%%%%%%%%%%%%%
\usepackage[hyphens]{url}

  \journal{Ecology Letters} % Sets Journal name


\usepackage{lineno} % add
  \linenumbers % turns line numbering on
\providecommand{\tightlist}{%
  \setlength{\itemsep}{0pt}\setlength{\parskip}{0pt}}

\usepackage{graphicx}
\usepackage{booktabs} % book-quality tables
%%%%%%%%%%%%%%%% end my additions to header

\usepackage[T1]{fontenc}
\usepackage{lmodern}
\usepackage{amssymb,amsmath}
\usepackage{ifxetex,ifluatex}
\usepackage{fixltx2e} % provides \textsubscript
% use upquote if available, for straight quotes in verbatim environments
\IfFileExists{upquote.sty}{\usepackage{upquote}}{}
\ifnum 0\ifxetex 1\fi\ifluatex 1\fi=0 % if pdftex
  \usepackage[utf8]{inputenc}
\else % if luatex or xelatex
  \usepackage{fontspec}
  \ifxetex
    \usepackage{xltxtra,xunicode}
  \fi
  \defaultfontfeatures{Mapping=tex-text,Scale=MatchLowercase}
  \newcommand{\euro}{€}
\fi
% use microtype if available
\IfFileExists{microtype.sty}{\usepackage{microtype}}{}
\usepackage[margin=1.1in]{geometry}
\bibliographystyle{elsarticle-harv}
\usepackage{longtable}
\ifxetex
  \usepackage[setpagesize=false, % page size defined by xetex
              unicode=false, % unicode breaks when used with xetex
              xetex]{hyperref}
\else
  \usepackage[unicode=true]{hyperref}
\fi
\hypersetup{breaklinks=true,
            bookmarks=true,
            pdfauthor={},
            pdftitle={Invasive mesopredator release--changes in feral cat density and behaviour following lethal fox control},
            colorlinks=false,
            urlcolor=blue,
            linkcolor=magenta,
            pdfborder={0 0 0}}
\urlstyle{same}  % don't use monospace font for urls

\setcounter{secnumdepth}{5}
% Pandoc toggle for numbering sections (defaults to be off)

% Pandoc citation processing

% Pandoc header
\usepackage{setspace}\doublespacing
\usepackage{float}
\floatplacement{figure}{H}



\begin{document}
\begin{frontmatter}

  \title{Invasive mesopredator release--changes in feral cat density and behaviour following lethal fox control}
    \author[UOM]{Matthew W. Rees\corref{1}}
   \ead{matt.wayne.rees@gmail.com} 
    \author[CEC]{Jack H. Pascoe}
  
    \author[CEC]{Mark Le Pla}
  
    \author[CEC]{Emma K. Birnbaum}
  
    \author[ARI]{Alan Robley}
  
    \author[UOM]{Brendan A. Wintle}
  
    \author[UOM]{Bronwyn A. Hradsky}
  
      \address[UOM]{Quantitative \& Applied Ecology Group, School of Ecosystem and Forest Science, The University of Melbourne, Parkville, VIC, Australia}
    \address[CEC]{Conservation Ecology Centre, Otway Lighthouse Rd, Cape Otway, VIC, Australia}
    \address[ARI]{Department of Environment, Land, Water and Planning, Arthur Rylah Institute for Environmental Research, Heidelberg, Australia}
      \cortext[1]{Corresponding Author}
  
  \begin{abstract}
  The mesopredator release hypothesis predicts increased densities of subordinate predators following apex predator decline. We conducted replicated field experiments across two regions with a simple predator guild comprising the introduced red fox \emph{Vulpes vulpes} and feral cat \emph{Felis catus}. We fit spatial mark-resight models to 63,560 camera-trap nights to characterise cat density as a function of fine scale variation in fox occurrence and landscape scale fox control. Cat density was highest where fox occurrence was lowest across both surveyed regions, suggesting integrated predator management may be necessary to protect native prey. However, cats did not respond consistently to broadscale fox control, owing to spatiotemporal variation in fox suppression effectiveness. Nonlinear models suggested behavioural avoidance of foxes was an effective strategy for cats at low, but not high fox occurrence probabilities. Mesopredator release can manifest in both behavioural and numerical changes, distorting inference when these processes are not separated.
  \end{abstract}
   \begin{keyword} camera trap; \emph{Felis catus}; invasive predator; interspecific competition; mesopredator release; population density; spatial capture-recapture; spatial mark-resight; species interactions; \emph{Vulpes vulpes}.\end{keyword}
 \end{frontmatter}

\parskip=12pt

\newpage

\hypertarget{introduction}{%
\section{INTRODUCTION}\label{introduction}}

Understanding species interactions is critical for effective invasive species management (Zavaleta \emph{et al.} 2001). When several invasive species co-occur, management actions that suppress the dominant invasive species may inadvertently benefit subordinate invasive species (Jackson 2015; Kuebbing \& Nuñez 2015). For example, the removal of a dominant invasive predator may increase the abundance of a subordinate invasive species due to a reduction in direct top-down pressure or indirect benefits from an increase in the availability of shared resources; this is often referred to as mesopredator or competitor release (Crooks \& Soulé 1999; Ruscoe \emph{et al.} 2011; Doherty \& Ritchie 2017). The release of subordinate invasive species, particularly predators, can have serious negative implications for native taxa and ecosystem function (Courchamp \emph{et al.} 1999; Ballari \emph{et al.} 2016). However, integrated invasive predator management is often far more costly and less feasible than single species control, and so it is important to identify when the extra cost is justified (Bode \emph{et al.} 2015).

Most knowledge of predator interactions stems from unreplicated `natural experiments' (e.g.~range contractions - Crooks \& Soulé 1999) or ad-hoc management interventions (e.g.~invasive species eradications - Rayner \emph{et al.} 2007). However, the occurrence, nature (positive or negative, direct or indirect) and strength of species interactions can vary among species assemblages, predation risk, environmental productivity, management regimes and other landscape contexts (Hastings 2001; Finke \& Denno 2004; Alston \emph{et al.} 2019). Replicating management programs in an experimental framework is logistically challenging, but important for understanding these complexities, discriminating between plausible hypotheses and producing generalisable results in order to inform effective pest management (Glen \& Dickman 2005; Christie \emph{et al.} 2019; Smith \emph{et al.} 2020).

Unbiased estimates of invasive predator density are important to infer their impacts on native prey and set meaningful control targets (Moseby \emph{et al.} 2019). Controversy around the mesopredator release hypothesis partially stems from the inability to distinguish behavioral from numerical population processes using traditional survey and modelling approaches (Hayward \emph{et al.} 2015; Stephens \emph{et al.} 2015). That is, the suppression of an apex predator may change the behaviour and the density of a mesopredator, both of which influence detection rates (Broadley \emph{et al.} 2019; Rogan \emph{et al.} 2019). This makes it difficult to interpret changes in unidentified counts or presence-absence records of mesopredators in relation to the presence, absence or abundance of apex predators, even within experimentally designed studies. In contrast, spatially explicit capture-recapture methods allow you to robustly estimate predator density by separating behavioural and observational processes from population density; however, they have seldom been used experimentally or to investigate multispecies interactions (although, see Forsyth \emph{et al.} 2019).

Predation by two invasive species, the red fox \emph{Vulpes vulpes} and feral cat \emph{Felis catus}, has played a major role in Australia's high rates of mammalian extinction (Woinarski et al.~2019). Integrated invasive predator management programs for these species are rare. Introduced red foxes (hereafter foxes) are far more commonly controlled than feral cats, as they are more susceptible to poison-baiting, have greater direct economic impacts and fewer legal impediments to their control (Reddiex \emph{et al.} 2007; McLeod \& Saunders 2014). Nonetheless, feral cats are one of the most widespread and damaging vertebrate species (Medina \emph{et al.} 2011; Doherty \& Ritchie 2017; Legge \emph{et al.} 2020). As foxes are larger-bodied (\textasciitilde2 kg difference) and have high dietary overlap with feral cats (Catling 1988; Short \emph{et al.} 1999; Glen \emph{et al.} 2011), the mesopredator release hypothesis predicts that the impacts of feral cats will increase as fox populations are suppressed (Soulé \emph{et al.} 1988). This is alarming because feral cats are extremely difficult to manage in open populations (Fisher \emph{et al.} 2015; Lazenby \emph{et al.} 2015).

Evidence that foxes suppress feral cats is inconclusive (Hunter \emph{et al.} 2018). In parts of Australia where the native apex mammalian predator (the dingo \emph{Canis familiaris}) is functionally extinct and introduced foxes are the largest terrestrial mammalian predator, four studies have observed an increase in feral cat detections following fox control (Risbey \emph{et al.} 2000; Marlow \emph{et al.} 2015; Stobo-Wilson \emph{et al.} 2020). However, two other studies in similar systems did not see any change (Towerton \emph{et al.} 2011; Molsher \emph{et al.} 2017). One study with spatial replication detected an increase at one site but not another (Davey \emph{et al.} 2006), and one study observed a decrease in feral cat activity (Claridge \emph{et al.} 2010). No previous studies have directly estimated feral cat density in response to fox control.

In this study, we experimentally investigated the role of introduced foxes in top-down suppression of feral cat density in two regions of south-eastern Australia. Our experiments were based around a replicated Control-Impact design in the region with long-term fox control, and a Before-After Control-Impact Paired Series (BACIPS) design in the region with newly implemented fox control. Foxes and feral cats are the only functional terrestrial mammalian predators in these regions, and each region included at least one area in which foxes were subject to continuous lethal poison-baiting (hereafter `impact landscape'), and a paired area where foxes were not controlled (hereafter `non-impact landscape'). This allowed a sharp focus on the interactions between the two invasive predators, across a gradient of apex predator (fox) occurrence and landscape productivity. In accordance with the mesopredator release hypothesis, we predicted that (1) feral cat density would be negatively correlated with spatial fox occurrence at a fine spatial scale, and (2) fox control would increase feral cat density at a broadscale. We based inference on direct estimates of feral cat density using spatially explicit mark-resight models.

\newpage

\hypertarget{materials-and-methods}{%
\section{MATERIALS AND METHODS}\label{materials-and-methods}}

\hypertarget{study-area}{%
\subsection{Study area}\label{study-area}}

We conducted our study across two regions of south-west Victoria, Australia (Fig. \ref{fig:map}). The native temperate forests in both regions are fragmented to varying degrees, primarily by livestock farming and tree plantations. Although once widespread, native dingoes are now absent throughout, and a native mesopredator, the tiger quoll \emph{Dasyurus maculatus} is long absent from the Glenelg region and extremely rare in the Otway Ranges (last sighted in 2014 despite extensive camera-trapping). The terrestrial mammalian predator guild is therefore depauperate, with the introduced fox and feral cat being the primary functional mammalian terrestrial predators; birds of prey and snakes are the only other carnivores present.

Our study landscapes in the Glenelg region, Gunditjmara country, are primarily lowland forest and heathy woodland. The area receives an average annual rainfall of 700 mm (``Bureau of meteorology'' 2021) and has gently undulating terrain. The region frequently experiences prescribed burns and wildfires, creating a mosaic of fire histories and vegetation complexity. Our study landscapes in the Otway region were in the western section of the Otway Ranges, Gadubanud country. Rainfall here is more than twice as high as the Glenelg region. The vegetation is a mosaic of shrubby wet forest and cool temperate rainforest, with our northern sites bordering on a large heathy woodland. This region rarely experiences fire and is nearly ten times more rugged than the Glenelg region (based on the terrain ruggedness index; Riley \emph{et al.} 1999).

\hypertarget{lethal-fox-control}{%
\subsection{Lethal fox control}\label{lethal-fox-control}}

Government land managers conduct ongoing targeted fox control for biodiversity conservation across broad sections of each region. Manufactured poison baits (FoxOff, Animal Control Technologies, Somerton) containing 3 mg of sodium mono-fluroacetate (1080) are buried at a depth of 12-15 cm at 1-km intervals along accessible forest tracks and roads (Fig. \ref{fig:map}). Different road densities across the two regions therefore result in variable poison-bait densities. In the Glenelg region, fox-baiting in the impact landscapes has been ongoing since October 2005, with baits checked and replaced fortnightly (Robley \emph{et al.} 2014). In the Otway region, fox-baiting commenced in the impact landscape in November 2017. Poison baits were replaced weekly for six weeks until December 2017, before changing to monthly bait replacement until July 2018. The fox control program then lapsed for approximately six months until December 2018, when monthly bait replacement resumed for the remainder of the study (Fig. S1).

\hypertarget{study-design-and-camera-trapping}{%
\subsection{Study design and camera-trapping}\label{study-design-and-camera-trapping}}

We designed experiments around the implementation of fox-baiting in each region. We simultaneously surveyed one impact and one non-impact landscape at a time. Each pair of impact and non-impact landscapes was chosen based on similarity in vegetation groups, with the aim of maintaining spatial independence with respect to predator daily movements.

In the Glenelg region, we used a replicated control-impact design to compare areas that have been poison-baited for foxes for more than 13 years with unbaited areas. We surveyed Cobboboonee National Park (impact) and Annya State Forest (non-impact) in January -- April 2018, then Mt Clay State Forest/Narrawong Flora Reserve (hereafter `Mt Clay'; impact) and Hotspur State Forest (non-impact) in April -- June 2018 (Fig. S1). Each grid was separated by at least 8 km, a distance very unlikely to be traversed regularly by these invasive predators (Hradsky \emph{et al.} 2017b).

In the Otway region, we undertook a before-after control-impact paired series (BACIPS) study to assess changes related to the introduction of the fox control program. We deployed camera-trap grids in an impact -- non-impact pair of landscapes from June to September in three years (2017, 2018, 2019), in the Great Otway National Park and Otway Forest Park (Fig. S1). Our first survey occurred approximately three months before fox-baiting began. The second survey was conducted six months after fox-baiting commenced, however poison bait replacement lapsed from near the beginning of the survey until nearly three months afterwards. Fox-baiting recommenced six months prior to the start of the final survey (Fig. S1). The impact and non-impact landscapes were at least 4.2 km apart through dense forest, a distance unlikely to be regularly traversed by these invasive predators, although possible (Hradsky \emph{et al.} 2017b). In this study, and a concurrent study which identified individual foxes through genetic sampling (M. Le Pla et al., in review), we found no evidence that either species of predator moved between the impact and non-impact landscapes.

In each of the six survey landscapes, we deployed a grid of camera-traps (67 -- 110 cameras; mean = 94), with sites spaced on average 448 m apart (range: 194 -- 770 m; Fig. \ref{fig:map}). At each site, we deployed a single Reconyx trail camera (Reconyx, Holmen, Wisconsin) with an infrared flash and temperature-in-motion detector on a tree, facing a lure of oil-absorbing cloth doused in tuna oil (Fig. S2). More information on the camera-trapping methods is provided in Section 1.2 of the Supporting Information. Overall, we deployed 938 functional camera-traps, which operated for an average of 68 days (range: 12 -- 93 days), totalling 62,415 trap nights (Table S1)

\hypertarget{individual-feral-cat-identification}{%
\subsection{Individual feral cat identification}\label{individual-feral-cat-identification}}

We sorted the cats into five categories based on their coat type: black, mackerel tabby, classic tabby, ginger and other (cats with multiple colour blends or other distinctive coats; Fig. S3). We did not attempt to identify any black cats, even the few with white splotches on their underside, as these markings could not always be seen. Within the other four coat categories, multiple observers identified individual cats based on their unique coat patterns where possible. Detailed information on this process is provided in Section 2 of the Supporting Information.

\hypertarget{spatial-fox-occurrence}{%
\subsection{Spatial fox occurrence}\label{spatial-fox-occurrence}}

We could not directly use fox presence-absence data from the camera-trap sites as an independent predictor of cat density, as spatial mark-resight models require covariate values for each grid cell in which density is estimated (see Section 2.6 below). We therefore used the fox presence-absence data for each camera-trap site to generate a spatially-interpolated layer of fox occurrence probability using binomial generalised additive mixed-effects models (Wood 2017). These models allow efficient nonlinear spatial estimates, although, assume perfect detection.

We built the fox occurrence models using the `mgcv' R-package (version 1.3.1; Wood 2011). We modelled fox presences and absences (response variable) across space (explanatory variable) separately for each region, with a duchon spline spatial smooth; these provide better predictions at the edge of surveyed space than other splines (Miller \& Wood 2014). In the Otway region, we included a random intercept for each camera-trap site to account for repeat sampling and did not share spatial information across the years (using a `by variable' smooth, with year as a factor). Differences in camera-trap deployment lengths were accounted for using a model offset.

\hypertarget{spatial-mark-resight-models-of-feral-cat-density}{%
\subsection{Spatial mark-resight models of feral cat density}\label{spatial-mark-resight-models-of-feral-cat-density}}

We used a spatial capture-recapture approach to estimate feral cat density (Borchers \& Efford 2008). These models consider counts of detections and non-detections of individual animals at trap locations (accounting for trap-specific survey effort) to estimate the location of each individual's activity centre. Spatial capture-recapture models commonly assume that individuals have approximately circular home ranges and spend the majority of time in the centre of their range (`activity centre'). The probability of observing an individual therefore decreases with distance from the activity centre. Two detectability parameters govern this process: \emph{g0}, the probability of detecting an individual per occasion in their activity centre, and \emph{sigma}: a spatial scale parameter which is related to the home range size. Multiple candidate shapes for this decline in detectability with distance from the activity centre (`detection function') can be modelled. We tested the half-normal and exponential detection functions in each region, carrying forward the function with the lowest Akaike's Information Criterion score adjusted for small sample size (AICc) for all subsequent model fitting (Burnham \& Anderson 2004).

Spatial capture-recapture models have been extended to consider situations where not all individuals in a population are identifiable (i.e.~some are unmarked) (Chandler \& Royle 2013). These spatial mark-resight models typically assume unmarked individuals to be a random sample of the population, sharing the same detection process as marked individuals, and so allow density to be estimated for the entire population. Spatial mark-resight models have four categories of sightings: (1) marked individuals - detections with known identities identified to the individual level at least once each session, (2) marked but unidentifiable individuals - detections of individuals with known identities, but for which the individual could not be determined in a given session (we had no detections in this category), (3) unmarked individuals - unidentified detections which definitely do not belong to the first two categories (in our study, this category comprised black cats) and (4) mark status uncertain - detections in which individuals cannot be identified and it is not clear whether the individual is of the marked or unmarked category.

We used closed population, sighting-only, spatial mark-resight models to estimate feral cat density using the maximum likelihood `secr' R-package (Efford 2021). Detections of the `mark status uncertain' category cannot be handled in the `secr' R package, we therefore added them to the unmarked detections rather than discard them (Moseby \emph{et al.} 2020). We condensed detection histories of each mark category to a binary presence-absence record per each camera-trap for a 24-hour length duration (`occasion'), beginning at midday. We ran separate models for each region and treated each camera-trap grid deployment as a `session'. We created a 4000-m buffer zone around each camera-trap location, and estimated feral cat density across this area, with a 200-m grid cell resolution. These habitat mask specifications were based on initial model trials and our knowledge of feral cat behaviour in these areas; the aim was to ensure density was estimated over a large enough area to encompass the activity centres of all feral cats exposed to our camera-traps, at a fine enough scale to minimise bias in density estimates.

For each region, we ran three sets of models: we (1) chose `base model' covariates to carry through to subsequent model sets, (2) tested fine-scale associations between spatial fox occurrence and feral cat density, and (3) experimentally evaluated the effect of fox control on feral cat density at the landscape scale. Each of these steps is described in more detail in the following paragraphs. We assessed the relative performance of models in each set using AICc score, with models within 2 delta AICc of the top-ranked model considered equally plausible, and strongly supported compared to other candidate models (Burnham \& Anderson 2004). We expected that foxes would impact both detectability parameters for feral cats concurrently, and so always used the same parameters for \emph{g0} and \emph{sigma} within a model (Efford \& Mowat, 2014).

Feral cat detectability may have decreased over each survey duration due to the lure scent fading, and density may differ across vegetation types. While we chose landscape pairs based on the similarity of ecological vegetation class groups (DELWP 2020), there were small differences in the relative proportion of each group across landscapes. To account for this, we first condensed vegetation groups into three categories for each region: cleared land, heathy woodlands, and either lowland forests (Glenelg region only) or wet forests (Otway region only). Detailed information on this process is provided in Section 6 of the Supporting Information. We included vegetation type as a habitat mask covariate. Camera-traps were lured with tuna oil scent, which likely faded over the survey duration and potentially reduced feral cat detectability (Rees \emph{et al.} 2019). To account for this, we fit a model where g0 had a linear trend over the survey duration for each camera-trap. We included year as a density covariate in the Otway region models to account for repeat sampling. We compared these models to a `null model' without the vegetation and linear time trend covariates. We carried these covariates through to all subsequent models if their respective model had an AICc score which was lower than the null model by more than two.

To test direct associations between feral cats and foxes, we used the spatial fox occurrence estimates (detailed in the previous section) as an explanatory variable. We tested three hypotheses for each region: (1) fox occurrence only affects feral cat density, (2) fox occurrence only affects feral cat detectability (both parameters), and (3) fox occurrence affects the density and the detectability of feral cats. We included year as a feral cat density covariate in the Otway region models to account for repeat sampling. Predator associations may be nonlinear (Johnson \& VanDerWal 2009), and so we fit these three models again using regression splines (generalised additive models called within the `secr' R-package). We assessed these three models per region relative to a null model which did not include fox occurrence covariates.

To investigate fox-baiting effects at a landscape scale, we first estimated density separately for each landscape (coded as a separate `session') and used AICc scores to choose between different specifications of the detectability parameters: (1) constant, (2) an effect of fox-baiting (but with separate effects in each spatial or temporal replicate), (3) an effect of pair in the Glenelg region (due to possible seasonal differences) or an effect of year in the Otway region, (4) the combination of fox-baiting and pair/year effects. We then took the top-ranked model and used visual inspection of the 95\% confidence intervals around the density estimates to evaluate the effects of fox control on feral cat density at a landscape level (ref). For the Glenelg region (replicated control-impact design), we assessed whether the difference in feral cat density between each pair of non-impact and impact landscapes overlapped zero. For the Otway region (BACIPS design), we assessed whether the difference between the non-impact and impact landscapes changed between years.

\newpage

\begin{figure}
\includegraphics[width=1\linewidth]{figs/fig1} \caption{Locations of our six study landscapes in south-west Victoria, Australia. The grids of camera-traps are denoted by white dots, the locations of fox poison-bait stations are denoted by smaller black dots. The Glenelg region is to the west and Otway region to the east. Native vegetation is indicated by dark green, with hill shading. Map tiles by Stamen Design, under CC BY 3.0, map data by OpenStreetMap, under CC BY SA.}\label{fig:map}
\end{figure}

\newpage

\hypertarget{results}{%
\section{RESULTS}\label{results}}

\hypertarget{fox-occurrence}{%
\subsection{Fox occurrence}\label{fox-occurrence}}

In the Glenelg region, foxes were detected at 55\% of camera-trap sites in the non-impact (unbaited) landscape and only 26\% of sites in the impact (baited) landscape, for the first replicate (Annya and Cobboboonee, respectively). For the second non-impact / impact pair, the difference was smaller, with foxes detected at 48\% and 35\% of sites respectively (Hotspur and Mt Clay). In the Otway region, naive fox occurrence rates increased approximately 3-fold over three years in the non-impact landscape , from 0.13 in 2017, to 0.27 in 2018 and 0.43 in 2019. In contrast, in the impact landscape, naive fox occurrence rates halved over the same period, from 0.36 of sites prior to poison-baiting in 2017, to 0.27 in 2018 (fox-baiting occurred prior to, but lapsed during this survey), and 0.17 in 2019 (where fox-baiting occurred during the survey and had occurred for six months prior).

Fine-scale spatial variation in fox occurrence was higher within the Glenelg region than the Otway region (Fig. \ref{fig:foxplot}). In fact, no spatial variation (i.e.~a completely smooth relationship with space) was predicted for the 2018 survey data in the Otway region (Fig. \ref{fig:foxplot}). The fox occurrence model for the Glenelg region had an adjusted R-squared value of 0.142 and reduced null deviance by 14.8\%; the model for the Otway region had an R-squared value of 0.24 and reduced null deviance by 27.8\%. Model summaries and spatial standard error estimates are presented in Section 5 of the Supporting Information.

\hypertarget{feral-cats-in-the-glenelg-region}{%
\subsection{Feral cats in the Glenelg region}\label{feral-cats-in-the-glenelg-region}}

In the Glenelg region, we recorded 222 feral cat detections from 26,792 camera-trap nights (Table S1). Sixty-five percent of detections were feral cats with unique natural markings; the remainder were black. We were able to identify 85\% of feral cat detections to the individual level; a total of 40 cats (9 -- 13 individuals per grid). The exponential detector function was supported over the half-normal function (Table S2). There was no support for a linear time trend on g0, nor for an impact of vegetation type on feral cat density (Table S3).

At a fine spatial scale, feral cat density was negatively and linearly associated with spatial fox occurrence (coefficient -0.34; standard error 0.15; Fig. \ref{fig:dcor}); feral cat detectability was not affected by fox occurrence. This model had an AICc which was 2.76 better than the null; models that included an impact of fox occurrence on detectability parameters performed worse than the null model based on AICc scores (Table S4). Regression splines did not change model predictions (Fig. \ref{fig:dcor}), with all nonlinear models performing consistently worse than their linear counterparts (Table S4).

At the landscape level, feral cat density was substantially higher in the first impact landscape than its paired non-impact landscape, but there was no difference in density for the second replicate (Fig. \ref{fig:diffg}). Feral cat detectability was not affected by fox baiting or landscape pair (Table S5).

\hypertarget{feral-cats-in-the-otway-region}{%
\subsection{Feral cats in the Otway region}\label{feral-cats-in-the-otway-region}}

In the Otway region, we recorded 1022 feral cat detections from 35,623 camera-trap nights (Table S1). Sixty-one percent of detections were feral cats with unique natural markings and we identified 86\% of these, a total of 98 cats (20 -- 30 per grid per year). The exponential detector function was strongly supported over the half-normal function (Table S6). There was no support for a linear time trend on g0, nor for an impact of vegetation type on feral cat density (Table S7).

Spatial fox occurrence was negatively correlated with feral cat density (linear model coefficient -0.26; SE 0.14; Fig. \ref{fig:dcor}). There was strong support for an effect of fox occurrence on feral cat detectability (Fig. \ref{fig:detcor}). Where fox occurrence was higher, feral cats were less detectable in their activity centres (i.e.~negative association with g0; coefficient -0.69; SE 0.21; Fig. \ref{fig:detcor}A) and ranged further (i.e.~positive association with sigma; coefficient 0.3; SE 0.09; Fig. \ref{fig:detcor}B). The linear model and nonlinear models were indistinguishable based on AICc scores (as was also the case for the other two model specifications), with at least 10.6 AICc scores lower than the null (Table S8). The shape of the nonlinear model differed from the linear model, although estimates were mostly contained within the linear model 95\% confidence intervals (Fig. \ref{fig:dcor}; Fig. \ref{fig:detcor}).

At a landscape-scale, feral cat density tended to be lower in the impact than non-impact landscape prior to baiting beginning (i.e.~in 2017), but the confidence interval for the difference overlapped zero (Fig. \ref{fig:diffo}A). This remained similar in 2018. In 2019, feral cat density declined in the non-impact landscape but remained constant in the impact landscape. This suggests that there was a relative increase in cat density as a result of fox baiting in 2019. However, there was substantial uncertainty around the change, reflected in the considerable overlap in confidence intervals for the difference in 2019 and previous years (Fig. \ref{fig:diffo}A). There was strong evidence that feral cat detectability differed across the three years, and in response to the fox-baiting (Table S9).

\newpage

\begin{figure}
\includegraphics[width=1\linewidth]{figs/fig2_600dpi} \caption{Predicted red fox Vulpes vulpes occurrence derived from generalised additive models within each impact (I) and associated non-impact (NI) landscape in the Glenelg (A) and Otway (B) regions, Australia. Predicted fox occurrence was used as a predictor of feral cat Felis catus density in the spatial mark-resight models.}\label{fig:foxplot}
\end{figure}

\newpage

\begin{figure}
\includegraphics[width=1\linewidth]{figs/foxD_600dpi} \caption{Linear (solid lines) and nonlinear (dashed lines) models of feral cat density and fox occurrence in the Glenelg and Otway regions, Australia. Shaded areas indicate 95\% confidence intervals.}\label{fig:dcor}
\end{figure}

\newpage

\begin{figure}
\includegraphics[width=1\linewidth]{figs/foxDet_otways_600dpi} \caption{Linear (solid lines) and nonlinear (dashed lines) models of feral cat detectability parameters in the Otway Ranges, Australia. The probability of detecting a feral cat in its activity centre per 24-hour occasion (g0) decreased with the probability of fox occurrence (B), while sigma (which is related to home range size) increased (C). Shaded areas indicate 95\% confidence intervals.}\label{fig:detcor}
\end{figure}

\newpage

\begin{figure}
\includegraphics[width=1\linewidth]{figs/glenelg_estimates_600dpi} \caption{Estimates of difference between feral cat *Felis catus* density in impact and non-impact landscapes (A) and predicted feral cat density estimates (B) in impact and non-impact landscapes in the Glenelg region, Australia. Poison baiting for foxes has been conducted in the impact landscapes for more than 13 years. Feral cat density was higher in the impact than non-impact landscape for the first replicate pair, but not for the second replicate pair. Error bars represent 95\% confidence intervals.}\label{fig:diffg}
\end{figure}

\newpage

\begin{figure}
\includegraphics[width=1\linewidth]{figs/otways_estimates_600dpi} \caption{Estimates of difference between impact and non-impact landscapes (A) and predicted density estimates (B) in the Otway region, Australia. There was no evidence that feral cat density differed between impact and non-impact landscapes in each year surveyed). In 2017, surveys were conducted approximately two months before fox control commenced in the impact landscape (red). Error bars represent 95\% confidence intervals.}\label{fig:diffo}
\end{figure}

\newpage

\hypertarget{discussion}{%
\section{DISCUSSION}\label{discussion}}

After individually identifying 137 feral cats, we found that feral cat density was highest where fox occurrence was lowest across both surveyed regions at a fine spatial scale. However, lethal fox control did not consistently increase feral cat density at a landscape scale. This is most likely due to the fox-baiting inconsistently suppressing fox occurrence, as well as our short sampling period in the Otway region post-baiting. Our study demonstrates the potential for mesopredator release following intensive and sustained fox control, but also highlights that it is unlikely to universally occur; fox-cat interactions are context and scale-dependent. More broadly, our study illustrates how correlative and experimental approaches provide complementary lines of evidence when investigating interactions between predator species, and the importance of disentangling changes in density from changes in detectability.

We were able to exploit a gradient in fox occurrence caused by using lethal fox control to investigate associations between feral cat density and fox occurrence at a fine spatial scale across two separate ecosystems. At this scale, we observed a similar negative association with feral cat density and fox occurrence in each ecosystem, although there was more uncertainty around the relationship in the Otway region. Across the gradient of fox occurrence, overall feral cat density was consistently higher in the Otway region than Glenelg, likely reflecting relatively high small mammal (i.e.~prey) abundance (Rees \emph{et al.} 2019) and lower fox occurrence at the landscape level.

There is contention around whether linear regression is appropriate for investigating correlations between different predator species, as subordinate predators may only be suppressed when apex predator abundance is high (Johnson \& VanDerWal 2009). We tested this by comparing AICc scores of linear and non-linear model specifications. There was no evidence of non-linear associations between foxes and feral cats in the Glenelg region, while linear and non-linear models performed equally well in the Otway region. Non-linear models in the Otway region predicted that feral cat density declined only in the mid-high range of fox occurrence , while behavioural changes were seen in the low-mid range of fox occurrence. Perhaps feral cats can successfully avoid foxes through behavioural change where foxes are rare, but this is ineffective where foxes are common. Fox avoidance strategies likely come at the expense of hunting success (Sih 1980), which may mean that they are untenable for feral cats in the Glenelg region where small mammal abundance is relatively low (Wilson \emph{et al.} 2010).

Where fox occurrence was higher in the Otway Ranges, feral cats were less detectable in their activity centres and ranged further (Fig. \ref{fig:detcor}). Low detectability is likely to correlate with fewer apex predator encounters, and has been observed in other predator interaction studies (e.g.~Lombardi \emph{et al.} 2017). Sigma scaling with fox occurrence probability supports observations made by Molsher \emph{et al.} (2017), and may reflect a direct avoidance strategy. Animal movement rates are expected to increase in response to unpredictable threats (Riotte-Lambert \& Matthiopoulos 2020). Alternatively, feral cats may consider foxes predictable and avoid locations they frequent, thus having to range further to cover the same home-range area. In forests similar to the Otway region, Buckmaster (2012) observed large `holes' in the home range of each GPS collared feral cat; they confirmed that this was not due to an absence of prey and so hypothesised that it could be due to apex predator avoidance. On the other hand, sigma scaling with fox occurrence probability may be an indirect effect - a symptom of the change in feral cat density. Feral cat density was lowest where fox occurrence was highest, and animals, including feral cats, tend to have larger home ranges at low densities (Bengsen \emph{et al.} 2016). Regardless of the mechanism, apex predator-dependent movement rates has serious implications for the interpretation of studies which compare the relative abundance (without disentangling behaviour and detectability from density) and spatial overlap of predator species (Efford \& Dawson 2012; Neilson \emph{et al.} 2018; Stewart \emph{et al.} 2018; Broadley \emph{et al.} 2019).

Feral cat density in the Glenelg region was similar across three out of the four landscapes surveyed (Fig. \ref{fig:diffg}). This is unsurprising given that landscape-wide fox occurrence estimates varied by less than 10\% across these three landscapes, despite a long-term fox baiting program at Mt Clay. In contrast, feral cat density was at least 67\% higher in the other impact (baited) landscape (Cobboboonee), and average fox occurrence was around 31\% lower. Fox baiting may be more effective at reducing fox occurrence at Cobboboonee where the camera-trap grid is largely surrounded by mostly fox-baited forest, than Mt Clay which is nearly entirely surrounded by unbaited farmland (Fig. \ref{fig:map}). Landscape feral cat density estimates are averaged across the entire camera-trap grid interior and surrounding 4 km buffer zone. Hradsky \emph{et al.} (2019) predicted that extending poison-baits 1 km into the adjacent farmland would more than halve fox density inside Mt Clay. Studies of fox-cat (and other predator-predator) interactions often rely solely on the presence of a management program as a proxy for lower apex predator abundance and distribution (e.g.~Hunter \emph{et al.} 2018). Our findings strongly indicate the need to directly measure the apex predator population in order to reliably interpret the response of subordinate species (Salo \emph{et al.} 2010).

We saw no evidence that feral cat density differed between impact and non-impact landscapes in the Otway region (Fig. \ref{fig:diffo}). There may not have been an effect because fox occurrence was already low prior to fox control commencing (Fig. \ref{fig:foxplot}). The non-linear correlative model predicted feral cat density to be unaffected by fox occurrence at such low levels (Fig. \ref{fig:dcor}). However, we did see a small increase in feral cat density in the impact landscape relative to the non-impact landscape during our final survey in 2019. Our short sampling period post-baiting may have not allowed enough time for changes to manifest in the adult population, as foxes possibly suppress cats through lowered breeding and recruitment rates (Ritchie \& Johnson 2009). Despite high feral cat density, the high proportion of unmarked cats in the Otway region, and complex detectability covariates reduced our ability to demonstrate statistical evidence for weak fox-baiting effects. Our surveys here provide an important baseline for which to compare future changes in predators and prey against, particularly once fox-baiting has occurred over a longer (and more consistent) time frame.

Our study has several limitations. It is unclear whether foxes may suppress feral cats directly through top-down control, or indirectly through competition for shared prey. Instead of responding directly to foxes, feral cats may have responded to fox-induced impacts on native prey (Stobo-Wilson \emph{et al.} 2020). However, vegetation type is a strong predictor of native prey occurrence within these regions (Hradsky \emph{et al.} 2017a), and we saw no evidence that vegetation type impacted feral cat density. Rather than avoidance or exclusion, negative correlation between foxes and feral cats may reflect differences in niche preference. However, this is less likely to be an issue for our study because foxes were artificially suppressed using lethal control. Uncertainty from our fox occurrence models was not propagated into the spatial mark-resight models, however, a full Bayesian integration of the fox occurrence analysis and the spatial mark-resight model is not yet implemented. The development of open population spatial mark-resight models would also improve parameter estimates for multi-season surveys.

By directly measuring apex predator population response to control and mesopredator density, we removed a key aspect of uncertainty that has hampered previous mesopredator release studies. The few studies which have estimated mesopredator density have mostly used live capture-rates to infer population density, without accounting for behavioural or detectability changes (Arjo \emph{et al.} 2007; Karki \emph{et al.} 2007; Thompson \& Gese 2007; Berger \emph{et al.} 2008; Jones \emph{et al.} 2008). Contention about mesopredator release has centred on such methods (Hayward \& Marlow 2014); as well unaccounted species interactions in complex predator guilds (Levi \& Wilmers 2012; Jachowski \emph{et al.} 2020). In contrast, our study tests the mesopredator release theory using a combined behavioural and numerical approach, in a simple system with only two carnivore species.

We saw behavioural and numerical responses consistent with a mesopredator release of feral cats, although numerical changes were only consistently observed at a finescale. This may explain why pest management that only targets foxes--one of the most prevalent conservation actions in Australia--does not consistently improve native prey persistence (Dexter \& Murray 2009; Wayne \emph{et al.} 2017; Lindenmayer \emph{et al.} 2018; Duncan \emph{et al.} 2020). However, more evidence is required to unequivocally prove that lethal fox control consistently causes increases in feral cat density. Regardless, there will be circumstances where targeted fox control is worthwhile because some native prey species are more susceptible to foxes than feral cats (Stobo-Wilson \emph{et al.} 2021). A more integrated approach to invasive predator management, whereby both foxes and feral cats are simultaneously or otherwise optimally controlled together could substantially improve biodiversity outcomes (Risbey \emph{et al.} 2000; Comer \emph{et al.} 2020). If this is not feasible, changes in invasive mesopredator density and the outcomes for prey species should be closely monitored with triggers for ceasing invasive apex predator control or commencing integrated management if single-species control proves counterproductive for conservation of threatened prey species.

\newpage

\hypertarget{acknowledgements}{%
\section{ACKNOWLEDGEMENTS}\label{acknowledgements}}

We acknowledge and pay respect to the Gadubanud and Gunditjmara peoples on whose traditional lands this study took place. Surveys were conducted under University of Melbourne Animal Ethics Committee approval 1714119 and Victorian Government Department of Environment, Land Water and Planning Research Permit 10008273. This experiment was conducted in cooperation with the Glenelg Ark (Department of Environment, Land Water and Planning) and Otway Ark (Parks Victoria) working groups. We are extremely grateful to our field assistants: Shauni Omond, Shayne Neal, Asitha Samarawickrama, Shelley Thompson, Erin Harris, Hannah Killian, Lani Watson, Mark Dorman, Jack Davis, Carl Roffey, Bruce Edley, Larissa Oliveira Gonçalves, Ben Lake, Chantelle Geissler, Aviya Naccarella, Emily Gronow, Harley England, David Pitts, Annie Hobby, Louise Falls, Thomas McKinnon, Jimmy Downie, Marney Hradsky, Stephanie Samson, Robin Sinclair, Asmaa Alhusainan, Kelly Forrester, Tammana Wadawani, Emily McColl-Gausden, Emily Gregg, Hannah Edwards, Adam Beck, Vishnu Memnon, Sandy Lu, Pia Lentini, Nick Golding, Emily McColl-Gausden, Nina Page, Maggie Campbell-Jones, Kyle Quinn and Jack Dickson. This manuscript was improved by comments from William Geary. Our study was generously supported by the Conservation Ecology Centre, the Victorian Government Department of Environment, Land Water and Planning, Arthur Rylah Institute for Environmental Research, Parks Victoria, the Australian Government's National Environmental Science Program through the Threatened Species Recovery Hub, and ARC Linkage Project LP170101134. MR also receives support from an Australian Government Research Training Program Scholarship.

\hypertarget{authors-contributions}{%
\section{AUTHORS' CONTRIBUTIONS}\label{authors-contributions}}

M.W.R, B.A.H, J.H.P, B.A.W and A.R conceived the ideas and designed the methodology; M.W.R, J.H.P, M.LP, E.K.B and B.A.H collected the data; M.W.R analysed the data and led the writing of the manuscript. All authors contributed critically to the drafts and gave final approval for publication.

\hypertarget{open-research}{%
\section{OPEN RESEARCH}\label{open-research}}

Raw data and code are on Github link xx.\\
Data will be deposited on the Dryad Digital Repository after acceptance.

\newpage

\hypertarget{references}{%
\section*{REFERENCES}\label{references}}
\addcontentsline{toc}{section}{REFERENCES}

\hypertarget{refs}{}
\leavevmode\hypertarget{ref-alston2019}{}%
Alston, J., Maitland, B., Brito, B., Esmaeili, S., Ford, A. \& Hays, B. \emph{et al.} (2019). Reciprocity in restoration ecology: When might large carnivore reintroduction restore ecosystems? \emph{Biological conservation}, 234, 82--89.

\leavevmode\hypertarget{ref-arjo2007}{}%
Arjo, W.M., Gese, E.M., Bennett, T.J. \& Kozlowski, A.J. (2007). Changes in kit fox--coyote--prey relationships in the Great Basin Desert, Utah. \emph{Western North American Naturalist}, 67, 389--401.

\leavevmode\hypertarget{ref-ballari2016}{}%
Ballari, S.A., Kuebbing, S.E. \& Nuñez, M.A. (2016). Potential problems of removing one invasive species at a time: A meta-analysis of the interactions between invasive vertebrates and unexpected effects of removal programs. \emph{PeerJ}, 4, e2029.

\leavevmode\hypertarget{ref-https:ux2fux2fdoi.orgux2f10.1111ux2fjzo.12290}{}%
Bengsen, A.J., Algar, D., Ballard, G., Buckmaster, T., Comer, S. \& Fleming, P.J.S. \emph{et al.} (2016). Feral cat home-range size varies predictably with landscape productivity and population density. \emph{Journal of Zoology}, 298, 112--120.

\leavevmode\hypertarget{ref-berger2008indirect}{}%
Berger, K.M., Gese, E.M. \& Berger, J. (2008). Indirect effects and traditional trophic cascades: A test involving wolves, coyotes, and pronghorn. \emph{Ecology}, 89, 818--828.

\leavevmode\hypertarget{ref-bode2015}{}%
Bode, M., Baker, C.M. \& Plein, M. (2015). Eradicating down the food chain: Optimal multispecies eradication schedules for a commonly encountered invaded island ecosystem. \emph{Journal of Applied Ecology}, 52, 571--579.

\leavevmode\hypertarget{ref-borchers2008}{}%
Borchers, D.L. \& Efford, M.G. (2008). Spatially explicit maximum likelihood methods for capture--recapture studies. \emph{Biometrics}, 64, 377--385.

\leavevmode\hypertarget{ref-broadley2019}{}%
Broadley, K., Burton, A.C., Avgar, T. \& Boutin, S. (2019). Density-dependent space use affects interpretation of camera trap detection rates. \emph{Ecology and evolution}, 9, 14031--14041.

\leavevmode\hypertarget{ref-2123-8123}{}%
Buckmaster, A.J. (2012). Ecology of the feral cat (felis catus) in the tall forests of far east gippsland. PhD thesis. University of Sydney.; School of Biological Sciences; University of Sydney.; School of Biological Sciences.

\leavevmode\hypertarget{ref-BOM2021}{}%
\emph{Bureau of meteorology}. (2021). \emph{Climate Data Online URL}. Available at: \url{http://www.bom.gov.au/climate/data/}. Last accessed.

\leavevmode\hypertarget{ref-burnham2004}{}%
Burnham, K.P. \& Anderson, D.R. (2004). Multimodel inference: Understanding aic and bic in model selection. \emph{Sociological methods \& research}, 33, 261--304.

\leavevmode\hypertarget{ref-catling1988}{}%
Catling, P. (1988). Similarities and contrasts in the diets of foxes, vulpes vulpes, and cats, felis catus, relative to fluctuating prey populations and drought. \emph{Wildlife Research}, 15, 307--317.

\leavevmode\hypertarget{ref-christie2019}{}%
Christie, A.P., Amano, T., Martin, P.A., Shackelford, G.E., Simmons, B.I. \& Sutherland, W.J. (2019). Simple study designs in ecology produce inaccurate estimates of biodiversity responses. \emph{Journal of Applied Ecology}, 56, 2742--2754.

\leavevmode\hypertarget{ref-claridge2010}{}%
Claridge, A.W., Cunningham, R.B., Catling, P.C. \& Reid, A.M. (2010). Trends in the activity levels of forest-dwelling vertebrate fauna against a background of intensive baiting for foxes. \emph{Forest Ecology and Management}, 260, 822--832.

\leavevmode\hypertarget{ref-comer2020integrating}{}%
Comer, S., Clausen, L., Cowen, S., Pinder, J., Thomas, A. \& Burbidge, A.H. \emph{et al.} (2020). Integrating feral cat (felis catus) control into landscape-scale introduced predator management to improve conservation prospects for threatened fauna: A case study from the south coast of western australia. \emph{Wildlife Research}, 47, 762--778.

\leavevmode\hypertarget{ref-courchamp1999}{}%
Courchamp, F., Langlais, M. \& Sugihara, G. (1999). Cats protecting birds: Modelling the mesopredator release effect. \emph{Journal of Animal Ecology}, 68, 282--292.

\leavevmode\hypertarget{ref-crooks1999}{}%
Crooks, K.R. \& Soulé, M.E. (1999). Mesopredator release and avifaunal extinctions in a fragmented system. \emph{Nature}, 400, 563--566.

\leavevmode\hypertarget{ref-davey2006}{}%
Davey, C., Sinclair, A., Pech, R.P., Arthur, A.D., Krebs, C.J. \& Newsome, A. \emph{et al.} (2006). Do exotic vertebrates structure the biota of australia? An experimental test in new south wales. \emph{Ecosystems}, 9, 992--1008.

\leavevmode\hypertarget{ref-delwp2020}{}%
DELWP. (2020). \emph{Bioregions and evc benchmarks}. Available at: \url{https://www.environment.vic.gov.au/biodiversity/bioregions-and-evc-benchmarks}. Last accessed.

\leavevmode\hypertarget{ref-dexter2009impact}{}%
Dexter, N. \& Murray, A. (2009). The impact of fox control on the relative abundance of forest mammals in east gippsland, victoria. \emph{Wildlife Research}, 36, 252--261.

\leavevmode\hypertarget{ref-doherty2017}{}%
Doherty, T.S. \& Ritchie, E.G. (2017). Stop jumping the gun: A call for evidence-based invasive predator management. \emph{Conservation Letters}, 10, 15--22.

\leavevmode\hypertarget{ref-duncan2020eruptive}{}%
Duncan, R.P., Dexter, N., Wayne, A. \& Hone, J. (2020). Eruptive dynamics are common in managed mammal populations. \emph{Ecology}, 101, e03175.

\leavevmode\hypertarget{ref-efford2021secr}{}%
Efford, M.G. (2021). \emph{Secr: Spatially explicit capture-recapture models. R package version 4.4.4}. Available at: \url{http://CRAN.R-project.org/package=secr}. Last accessed.

\leavevmode\hypertarget{ref-efford2012}{}%
Efford, M.G. \& Dawson, D.K. (2012). Occupancy in continuous habitat. \emph{Ecosphere}, 3, 1--15.

\leavevmode\hypertarget{ref-finke2004}{}%
Finke, D.L. \& Denno, R.F. (2004). Predator diversity dampens trophic cascades. \emph{Nature}, 429, 407--410.

\leavevmode\hypertarget{ref-fisher2015}{}%
Fisher, P., Algar, D., Murphy, E., Johnston, M. \& Eason, C. (2015). How does cat behaviour influence the development and implementation of monitoring techniques and lethal control methods for feral cats? \emph{Applied Animal Behaviour Science}, 173, 88--96.

\leavevmode\hypertarget{ref-forsyth2019}{}%
Forsyth, D.M., Ramsey, D.S. \& Woodford, L.P. (2019). Estimating abundances, densities, and interspecific associations in a carnivore community. \emph{The Journal of Wildlife Management}, 83, 1090--1102.

\leavevmode\hypertarget{ref-glen2011}{}%
Glen, A., Pennay, M., Dickman, C., Wintle, B. \& Firestone, K. (2011). Diets of sympatric native and introduced carnivores in the barrington tops, eastern australia. \emph{Austral Ecology}, 36, 290--296.

\leavevmode\hypertarget{ref-glen2005}{}%
Glen, A.S. \& Dickman, C.R. (2005). Complex interactions among mammalian carnivores in australia, and their implications for wildlife management. \emph{Biological reviews}, 80, 387--401.

\leavevmode\hypertarget{ref-hastings2001}{}%
Hastings, A. (2001). Transient dynamics and persistence of ecological systems. \emph{Ecology Letters}, 4, 215--220.

\leavevmode\hypertarget{ref-hayward2015}{}%
Hayward, M.W., Boitani, L., Burrows, N.D., Funston, P.J., Karanth, K.U. \& MacKenzie, D.I. \emph{et al.} (2015). Ecologists need robust survey designs, sampling and analytical methods. \emph{Journal of Applied Ecology}, 52, 286--290.

\leavevmode\hypertarget{ref-https:ux2fux2fdoi.orgux2f10.1111ux2f1365-2664.12250}{}%
Hayward, M.W. \& Marlow, N. (2014). Will dingoes really conserve wildlife and can our methods tell? \emph{Journal of Applied Ecology}, 51, 835--838.

\leavevmode\hypertarget{ref-hradsky2019foxnet}{}%
Hradsky, B.A., Kelly, L.T., Robley, A. \& Wintle, B.A. (2019). FoxNet: An individual-based model framework to support management of an invasive predator, the red fox. \emph{Journal of Applied Ecology}, 56, 1460--1470.

\leavevmode\hypertarget{ref-hradsky2017bayesian}{}%
Hradsky, B.A., Penman, T.D., Ababei, D., Hanea, A., Ritchie, E.G. \& York, A. \emph{et al.} (2017a). Bayesian networks elucidate interactions between fire and other drivers of terrestrial fauna distributions. \emph{Ecosphere}, 8, e01926.

\leavevmode\hypertarget{ref-hradsky2017human}{}%
Hradsky, B.A., Robley, A., Alexander, R., Ritchie, E.G., York, A. \& Di Stefano, J. (2017b). Human-modified habitats facilitate forest-dwelling populations of an invasive predator, vulpes vulpes. \emph{Scientific Reports}, 7, 1--12.

\leavevmode\hypertarget{ref-hunter2018}{}%
Hunter, D.O., Lagisz, M., Leo, V., Nakagawa, S. \& Letnic, M. (2018). Not all predators are equal: A continent-scale analysis of the effects of predator control on australian mammals. \emph{Mammal Review}, 48, 108--122.

\leavevmode\hypertarget{ref-https:ux2fux2fdoi.orgux2f10.1111ux2fmam.12207}{}%
Jachowski, D.S., Butler, A., Eng, R.Y.Y., Gigliotti, L., Harris, S. \& Williams, A. (2020). Identifying mesopredator release in multi-predator systems: A review of evidence from north america. \emph{Mammal Review}, 50, 367--381.

\leavevmode\hypertarget{ref-jackson2015}{}%
Jackson, M.C. (2015). Interactions among multiple invasive animals. \emph{Ecology}, 96, 2035--2041.

\leavevmode\hypertarget{ref-https:ux2fux2fdoi.orgux2f10.1111ux2fj.1365-2664.2009.01650.x}{}%
Johnson, C.N. \& VanDerWal, J. (2009). Evidence that dingoes limit abundance of a mesopredator in eastern australian forests. \emph{Journal of Applied Ecology}, 46, 641--646.

\leavevmode\hypertarget{ref-jones2008sudden}{}%
Jones, K.L., Van Vuren, D.H. \& Crooks, K.R. (2008). Sudden Increase in a Rare Endemic Carnivore: Ecology of the Island Spotted Skunk. \emph{Journal of Mammalogy}, 89, 75--86.

\leavevmode\hypertarget{ref-karki2007}{}%
Karki, S.M., Gese, E.M. \& Klavetter, M.L. (2007). Effects of coyote population reduction on swift fox demographics in southeastern colorado. \emph{The Journal of Wildlife Management}, 71, 2707--2718.

\leavevmode\hypertarget{ref-kuebbing2015}{}%
Kuebbing, S.E. \& Nuñez, M.A. (2015). Negative, neutral, and positive interactions among nonnative plants: Patterns, processes, and management implications. \emph{Global Change Biology}, 21, 926--934.

\leavevmode\hypertarget{ref-lazenby2015}{}%
Lazenby, B.T., Mooney, N.J. \& Dickman, C.R. (2015). Effects of low-level culling of feral cats in open populations: A case study from the forests of southern tasmania. \emph{Wildlife Research}, 41, 407--420.

\leavevmode\hypertarget{ref-legge2020}{}%
Legge, S., Taggart, P.L., Dickman, C.R., Read, J.L. \& Woinarski, J.C. (2020). \emph{Wildlife Research}, 47, 731--746.

\leavevmode\hypertarget{ref-https:ux2fux2fdoi.orgux2f10.1890ux2f11-0165.1}{}%
Levi, T. \& Wilmers, C.C. (2012). Wolves--coyotes--foxes: A cascade among carnivores. \emph{Ecology}, 93, 921--929.

\leavevmode\hypertarget{ref-LINDENMAYER2018279}{}%
Lindenmayer, D.B., Wood, J., MacGregor, C., Foster, C., Scheele, B. \& Tulloch, A. \emph{et al.} (2018). Conservation conundrums and the challenges of managing unexplained declines of multiple species. \emph{Biological Conservation}, 221, 279--292.

\leavevmode\hypertarget{ref-lombardi2017coyote}{}%
Lombardi, J.V., Comer, C.E., Scognamillo, D.G. \& Conway, W.C. (2017). Coyote, fox, and bobcat response to anthropogenic and natural landscape features in a small urban area. \emph{Urban Ecosystems}, 20, 1239--1248.

\leavevmode\hypertarget{ref-marlow2015}{}%
Marlow, N.J., Thomas, N.D., Williams, A.A., Macmahon, B., Lawson, J. \& Hitchen, Y. \emph{et al.} (2015). Cats (felis catus) are more abundant and are the dominant predator of woylies (bettongia penicillata) after sustained fox (vulpes vulpes) control. \emph{Australian Journal of Zoology}, 63, 18--27.

\leavevmode\hypertarget{ref-mcleod2014}{}%
McLeod, S.R. \& Saunders, G. (2014). Fertility control is much less effective than lethal baiting for controlling foxes. \emph{Ecological Modelling}, 273, 1--10.

\leavevmode\hypertarget{ref-medina2011}{}%
Medina, F.M., Bonnaud, E., Vidal, E., Tershy, B.R., Zavaleta, E.S. \& Josh Donlan, C. \emph{et al.} (2011). A global review of the impacts of invasive cats on island endangered vertebrates. \emph{Global Change Biology}, 17, 3503--3510.

\leavevmode\hypertarget{ref-miller2014}{}%
Miller, D.L. \& Wood, S.N. (2014). Finite area smoothing with generalized distance splines. \emph{Environmental and ecological statistics}, 21, 715--731.

\leavevmode\hypertarget{ref-molsher2017}{}%
Molsher, R., Newsome, A.E., Newsome, T.M. \& Dickman, C.R. (2017). Mesopredator management: Effects of red fox control on the abundance, diet and use of space by feral cats. \emph{PLoS One}, 12, e0168460.

\leavevmode\hypertarget{ref-moseby2019}{}%
Moseby, K.E., Letnic, M., Blumstein, D.T. \& West, R. (2019). Understanding predator densities for successful co-existence of alien predators and threatened prey. \emph{Austral Ecology}, 44, 409--419.

\leavevmode\hypertarget{ref-moseby2020effectiveness}{}%
Moseby, K., McGregor, H. \& Read, J. (2020). Effectiveness of the felixer grooming trap for the control of feral cats: A field trial in arid south australia. \emph{Wildlife Research}, 47, 599--609.

\leavevmode\hypertarget{ref-https:ux2fux2fdoi.orgux2f10.1002ux2fecs2.2092}{}%
Neilson, E.W., Avgar, T., Burton, A.C., Broadley, K. \& Boutin, S. (2018). Animal movement affects interpretation of occupancy models from camera-trap surveys of unmarked animals. \emph{Ecosphere}, 9, e02092.

\leavevmode\hypertarget{ref-rayner2007}{}%
Rayner, M.J., Hauber, M.E., Imber, M.J., Stamp, R.K. \& Clout, M.N. (2007). Spatial heterogeneity of mesopredator release within an oceanic island system. \emph{Proceedings of the National Academy of Sciences}, 104, 20862--20865.

\leavevmode\hypertarget{ref-reddiex2007}{}%
Reddiex, B., Forsyth, D.M., McDonald-Madden, E., Einoder, L.D., Griffioen, P.A. \& Chick, R.R. \emph{et al.} (2007). Control of pest mammals for biodiversity protection in australia. I. Patterns of control and monitoring. \emph{Wildlife Research}, 33, 691--709.

\leavevmode\hypertarget{ref-rees2019}{}%
Rees, M., Pascoe, J., Wintle, B., Le Pla, M., Birnbaum, E. \& Hradsky, B. (2019). Unexpectedly high densities of feral cats in a rugged temperate forest. \emph{Biological Conservation}, 239, 108287.

\leavevmode\hypertarget{ref-riley1999}{}%
Riley, S.J., DeGloria, S.D. \& Elliot, R. (1999). Index that quantifies topographic heterogeneity. \emph{intermountain Journal of sciences}, 5, 23--27.

\leavevmode\hypertarget{ref-RIOTTELAMBERT2020163}{}%
Riotte-Lambert, L. \& Matthiopoulos, J. (2020). Environmental predictability as a cause and consequence of animal movement. \emph{Trends in Ecology \& Evolution}, 35, 163--174.

\leavevmode\hypertarget{ref-risbey2000}{}%
Risbey, D.A., Calver, M.C., Short, J., Bradley, J.S. \& Wright, I.W. (2000). The impact of cats and foxes on the small vertebrate fauna of heirisson prong, western australia. II. A field experiment. \emph{Wildlife Research}, 27, 223--235.

\leavevmode\hypertarget{ref-ritchie2009predator}{}%
Ritchie, E.G. \& Johnson, C.N. (2009). Predator interactions, mesopredator release and biodiversity conservation. \emph{Ecology Letters}, 12, 982--998.

\leavevmode\hypertarget{ref-robley2014}{}%
Robley, A., Gormley, A.M., Forsyth, D.M. \& Triggs, B. (2014). Long-term and large-scale control of the introduced red fox increases native mammal occupancy in australian forests. \emph{Biological Conservation}, 180, 262--269.

\leavevmode\hypertarget{ref-rogan2019}{}%
Rogan, M.S., Balme, G.A., Distiller, G., Pitman, R.T., Broadfield, J. \& Mann, G.K. \emph{et al.} (2019). The influence of movement on the occupancy--density relationship at small spatial scales. \emph{Ecosphere}, 10, e02807.

\leavevmode\hypertarget{ref-ruscoe2011}{}%
Ruscoe, W.A., Ramsey, D.S., Pech, R.P., Sweetapple, P.J., Yockney, I. \& Barron, M.C. \emph{et al.} (2011). Unexpected consequences of control: Competitive vs. Predator release in a four-species assemblage of invasive mammals. \emph{Ecology Letters}, 14, 1035--1042.

\leavevmode\hypertarget{ref-https:ux2fux2fdoi.orgux2f10.1890ux2f09-1260.1}{}%
Salo, P., Banks, P.B., Dickman, C.R. \& Korpimäki, E. (2010). Predator manipulation experiments: Impacts on populations of terrestrial vertebrate prey. \emph{Ecological Monographs}, 80, 531--546.

\leavevmode\hypertarget{ref-short1999}{}%
Short, J., Calver, M.C. \& Risbey, D.A. (1999). The impact of cats and foxes on the small vertebrate fauna of heirisson prong, western australia. I. Exploring potential impact using diet analysis. \emph{Wildlife Research}, 26, 621--630.

\leavevmode\hypertarget{ref-SIH1041}{}%
Sih, A. (1980). Optimal behavior: Can foragers balance two conflicting demands? \emph{Science}, 210, 1041--1043.

\leavevmode\hypertarget{ref-smith2020}{}%
Smith, J.A., Suraci, J.P., Hunter, J.S., Gaynor, K.M., Keller, C.B. \& Palmer, M.S. \emph{et al.} (2020). Zooming in on mechanistic predator--prey ecology: Integrating camera traps with experimental methods to reveal the drivers of ecological interactions. \emph{Journal of Animal Ecology}, 89, 1997--2012.

\leavevmode\hypertarget{ref-soule1988}{}%
Soulé, M.E., Bolger, D.T., Alberts, A.C., Wrights, J., Sorice, M. \& Hill, S. (1988). Reconstructed dynamics of rapid extinctions of chaparral-requiring birds in urban habitat islands. \emph{Conservation Biology}, 2, 75--92.

\leavevmode\hypertarget{ref-stephens2015}{}%
Stephens, P.A., Pettorelli, N., Barlow, J., Whittingham, M.J. \& Cadotte, M.W. (2015). Management by proxy? The use of indices in applied ecology.

\leavevmode\hypertarget{ref-stewart2018}{}%
Stewart, F.E.C., Fisher, J.T., Burton, A.C. \& Volpe, J.P. (2018). Species occurrence data reflect the magnitude of animal movements better than the proximity of animal space use. \emph{Ecosphere}, 9, e02112.

\leavevmode\hypertarget{ref-stobo2020management}{}%
Stobo-Wilson, A.M., Brandle, R., Johnson, C.N. \& Jones, M.E. (2020). Management of invasive mesopredators in the flinders ranges, south australia: Effectiveness and implications. \emph{Wildlife Research}, 47, 720--730.

\leavevmode\hypertarget{ref-stobo2021reptiles}{}%
Stobo-Wilson, A.M., Murphy, B.P., Legge, S.M., Chapple, D.G., Crawford, H.M. \& Dawson, S.J. \emph{et al.} (2021). Reptiles as food: Predation of australian reptiles by introduced red foxes compounds and complements predation by cats. \emph{Wildlife Research}.

\leavevmode\hypertarget{ref-thompson2007food}{}%
Thompson, C.M. \& Gese, E.M. (2007). Food webs and intraguild predation: Community interactions of a native mesocarnivore. \emph{Ecology}, 88, 334--346.

\leavevmode\hypertarget{ref-towerton2011}{}%
Towerton, A.L., Penman, T.D., Kavanagh, R.P. \& Dickman, C.R. (2011). Detecting pest and prey responses to fox control across the landscape using remote cameras. \emph{Wildlife Research}, 38, 208--220.

\leavevmode\hypertarget{ref-wayne2017recoveries}{}%
Wayne, A.F., Maxwell, M.A., Ward, C.G., Wayne, J.C., Vellios, C.V. \& Wilson, I.J. (2017). Recoveries and cascading declines of native mammals associated with control of an introduced predator. \emph{Journal of Mammalogy}, 98, 489--501.

\leavevmode\hypertarget{ref-wilson2010prey}{}%
Wilson, R.R., Blankenship, T.L., Hooten, M.B. \& Shivik, J.A. (2010). Prey-mediated avoidance of an intraguild predator by its intraguild prey. \emph{Oecologia}, 164, 921--929.

\leavevmode\hypertarget{ref-wood2011}{}%
Wood, S.N. (2011). Fast stable restricted maximum likelihood and marginal likelihood estimation of semiparametric generalized linear models. \emph{Journal of the Royal Statistical Society: Series B (Statistical Methodology)}, 73, 3--36.

\leavevmode\hypertarget{ref-wood2017}{}%
Wood, S.N. (2017). \emph{Generalized additive models: An introduction with r}. CRC press.

\leavevmode\hypertarget{ref-zavaleta2001}{}%
Zavaleta, E.S., Hobbs, R.J. \& Mooney, H.A. (2001). Viewing invasive species removal in a whole-ecosystem context. \emph{Trends in Ecology \& Evolution}, 16, 454--459.


\end{document}


